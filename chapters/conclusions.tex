The isolation guarantees offered by virtualization technologies were exploited in many different ways to add a layer of security to a running guest operating system. This work proposed a system developed using a \emph{hybrid} approach, implementing a part of it both in the hypervisor and guest kernel. This allowed to introduce a hypervisor-protected guest agent inside the running virtual machine, capable to retrieve data from the guest kernel and to make them being available to the hypervisor. All information retrieved inside the guest are passed to the hypervisor by means of a secured paravirtualized channel, used to implement a hypercall-based API between the guest and the hypervisor. Some of the most important hypercall that were implemented were used to protect the guest agent or to enforce security measures on important kernel data structures. In particular, these hypercalls allowed to implement the Memory Protection Hypercall, the Save \& Compare and the Save \& Reload approach. Then, it was shown how the guest agent, which is a LKM inside the guest, can retrieve the address of any of the symbols of the running kernel using the double kprobe technique. This was useful to let the guest agent indicate to the hypervisor where important objects reside in memory and it is the means through which the semantic gap problem was solved. It was also shown, through the implementation of an attack scenario, how the system tries to detect malicious activity inside the guest. In particular, this was possible by implementing a monitoring agent capable to retrieve the list of processes, their open files, and their network connections. 
\par
Future work will try to build on top of the proposed system. In particular, the proposed system can be used as a framework to collect new data from the running guest kernel and to build another software tool that can examine them in real-time. This new part of the system could potentially run inside the hypervisor or even in another virtual machine, to enhance its security even more. Then, the next step is to actually block the attack once it is detected. This is achievable by killing guest processes when malicious activities are detected. To actually improve the detection part, the system can implement a rule engine: data will be checked against a set of rules, that are made up of a condition and an action. Whenever the condition is met, the corresponding action will be performed inside the guest by the guest agent. In the near future, Intel will also release the Hypervisor-Managed Linear Address Translation, HLAT for short, documented in Chapter 6 of the Instruction Set Extensions and Future Features Programming Reference \cite{hlat}. HLAT will allow the hypervisor to enforce guest virtual address translations. The objective of this new hardware feature is twofold: 
\begin{enumerate}
    \item It is used to ensure page table entries integrity
    \item It is used to ensure integrity of code and data of the running kernel, without using the EPT bits
\end{enumerate}
The hypervisor will take the ownership of a part of the guest page tables. It will be able to use their bits directly to enforce specific kind of memory accesses. By taking the ownership, the guest will no longer be able to modify them and that is why this hardware feature allows to implement both point 1 and 2. The proposed system could try to use this new Intel extension to further improve itself in the future.




