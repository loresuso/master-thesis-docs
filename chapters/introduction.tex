As stated by NIST (National Institute of Standards and Technology), 
\say{\emph{cloud computing is a model for enabling ubiquitous, convenient, on-demand network access to a shared pool of configurable computing resources (e.g., networks, servers, storage, applications and services) that can be rapidly provisioned and released with minimal management effort or service provider interaction}} \cite{nistcloud}.
Virtualization is one of the enabling technologies of cloud computing and, in part, it helped cloud computing to become a reality. Indeed, virtual machines can be easily configured and rapidly deployed, like discussed in the definition. Furthermore, these tasks are often automated, opening the possibility of implementing dynamic and on-demand provisioning and to scale horizontally. Thanks to virtualization, utilization of the physical infrastructure is significantly improved by sharing physical resources and allowing virtual servers to move from one physical machine to another. 
\par
One of the key tasks of a hypervisor is to keep guest virtual machines isolated, with respect to other virtual machines and the host, and to precisely control the usage of the underlying physical hardware. The isolation guarantees offered by the hypervisor are of particular interest also from a security standpoint, since the hypervisor can control the execution of a guest and analyze its state at the hardware level. 
\par
This thesis attempts to explore ways to use existing virtualization technologies to add a layer of security to a guest OS, possibly detecting and preventing attacks to the guest OS itself or to applications running on it, by executing monitoring code in a secure and stealthy way. 

\section{Problem statement}
A system to observe and monitor the guest's state can be deployed both in the guest itself or in the hypervisor. Both these two approaches have advantages and disadvantages. 
\par
Considering the \emph{in-guest} approach, the monitoring system is placed in the same OS that it is monitoring. In fact, it is visible to an attacker and this is one of the main drawbacks of this approach. The only line of defense between the monitoring system and the attacker is solely offered by the operating system. If an attacker manages to escalate privileges by exploiting some vulnerability (or a chain of vulnerabilities), the guest OS cannot protect the monitoring agent anymore and it becomes part of the attack surface. As an advantage, instead, given that the system is inside the guest OS, it has access to all the semantic information, such as files, kernel data structures, processes and much more, and can readily use this information to monitor the state of the virtual machine. 
\par
For what concerns the \emph{in-hypervisor} approach, the monitoring system has access to the hardware state of the guest. For instance, it can inspect its memory and its registers. Since the monitoring agent is outside the virtual machine, the isolation provided by the virtualization layer ensure that an attacker gaining privileges in the guest cannot tamper with the system anymore, assuming that the software and the underlying hardware that allows the implementation of the hypervisor are safe. From a security point of view, this is of paramount importance, because the system can proceed with its work regardless the state of the guest. However, this approach lacks the semantic information that is available in the former approach. For instance, it is true that memory can be accessed, but in order to extract meaningful information from it, these kind of systems have to interpret bytes into a higher level semantic. In literature, this problem is often referred to as the \emph{semantic gap}. The isolation offered by the virtualization layer is very attractive for the implementation of these kinds of systems, and that is why this approach was used by many researchers, leading to the birth of a set of techniques referred to as Virtual Machine Introspection. 
\par 
This thesis will attempt to find new and creative ways for executing monitoring code using a hybrid approach, trying to combine the advantages of the two previously discussed approaches.

\section{Objectives of monitoring agents}
Generally speaking, monitoring agents can be used to implement intrusion detection and prevention systems to detect and block attacks. Monitoring the activities of guest systems can also be used to implement:
\begin{itemize}
    \item forensics tools and malware detection, to derive an in depth view of an attack
    \item virtual machine based honeypots, as an alternative way to study malware and different kind of attacks
    \item system recovery, to restore the system in a safe state after an attack
\end{itemize}
Furthermore, monitoring a guest system is a useful feature to be offered to users by cloud providers. This can help to not only protect users, but also the infrastructures of cloud providers, since many hypervisor related attacks often start from a compromised virtual machine. 

\section{Related work}
As noticed earlier, the literature offers a wide variety of systems and projects, which mainly fall in the two categories of \emph{in-guest} and \emph{in-hypervisor} approaches. This section discusses some of the features of two relevant projects, namely, the Linux Kernel Runtime Guard and LibVMI.
\par
The Linux Kernel Runtime Guard \cite{lkrg} is an open source project carried on by OpenWall, a company whose leader is Solar Designer, a famous security researcher. This project represents the in-guest approach, even if it was intended to run not only in virtualized environments. As the name suggests, the LKRG is shipped as a kernel module whose aim is to "post-detect and hopefully promptly respond to unauthorized modifications to the running Linux kernel (integrity checking) or to credentials such as user IDs of the running processes (exploit detection)". Whenever the attacker successfully upgrades its credential through a kernel attack, any further attempt to use them is denied before the kernel would grant the access, e.g opening a file with those new credentials. When such behaviour is detected, the process is killed. LKRG also tries to provide system integrity by protecting important kernel objects such as the .text section containing kernel code and modules, critical kernel variables and CPU registers. The default action is to make the kernel panic, since any milder response would be not effective. It is interesting to notice that developers of this project are aware that this system is \emph{bypassable by design}, but still raises the bar of difficulty for many kinds of already or not yet known type of attacks. The reason why it is bypassable, though, was already explained: the system is visible to attackers and it is difficult to protect it from the same privilege level. 
\par 
LibVMI \cite{libvmi}, instead, is the de facto standard for the in-hypervisor approach. In general, most of the in-hypervisor based approaches are related to Virtual Machine Introspection techniques. The main reason for using these kinds of techniques is because they allow:
\begin{itemize}
    \item minimum performance impact on the virtual machine, since the monitoring software is always running outside the VM
    \item minimum modifications to the hypervisor, which helps to put in place the monitoring system while not increasing the complexity of the hypervisor, thus not enlarging the attack surface
    \item little modifications to the guest operating system, to let the system be deployable on a wide variety of systems with little effort
    \item security of the monitoring system, guaranteed by the isolation features already provided by virtualization.
\end{itemize}
One of the main techniques used to implement those kinds of system is the memory introspection, and this is what also LibVMI tries to accomplish. Main memory contains several important data structures such as process descriptors, loadable kernel modules, page tables, and others. For this reason, LibVMI is focused on reading and writing memory from virtual machines. For convienence, it also provides functions for accessing CPU registers, pausing and unpausing a VM, printing binary data, and hooking on hardware events. The library also offers Python bindings and can be used together with Volatility, a pretty famous memory forensic tool. However, the preliminary work of any VMI tool is to first bridge the semantic gap, i.e extract meaningful information from memory. This is the most difficult task that this kind of systems have to perform and this is why researchers are trying to find alternative ways to solve it. 
